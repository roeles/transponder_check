%\renewcommand*{\pagedeclaration}[1]
%{
%	\unskip
%	, page \hyperpage{#1}
%}

%Derivative
\newcommand{\derivative}[2]
{
	\frac{\partial #1}{\partial #2}
}

%Scientific notation
\newcommand{\e}[1]{\ensuremath{\times 10^{#1}}}

%Checklist commands
\newcommand{\itemcheck}[1]
{
        \item[\Squarepipe]{#1}
}

\newcommand{\itemchecked}[2]
{
        \item[\Checkedbox]{#1}
        \emph{#2}
}

\newcommand{\itemfailed}[2]
{
        \item[\Info]{#1}
        \emph{#2}
}

%These are derived from ISO-1151
\newcommand{\angleofattack}
{
	\alpha
}

\newcommand{\angleofsideslip}
{
	\beta
}

\newcommand{\azimuth}
{
	\Psi
}

\newcommand{\inclination}
{
	\Theta
}

\newcommand{\bankangle}
{
	\Phi
}

\newcommand{\angularvelocity}
{
	\Omega
}

\newcommand{\mach}
{
	Ma
}

%Quaternions
\newcommand{\quaternion}[4] {
\begin{bmatrix}
#4 \\
#1 \\
#2 \\
#3 \\
\end{bmatrix}
}

%Statistics
\newcommand{\stdev}
{
	\sigma
}

\newcommand{\variance}
{
	\sigma^2
}

%Misc
\newcommand{\nomen}[3]
{
	\nomenclature{#1}{#2 \hfill\makebox[8em]{#3\hfill}}
	Where #1 is the #2 (#3). \\
}

\newcommand{\myquote}[1]
{
	\textit{#1}
}

\newcommand{\chapterref}[1]
{
	Chapter \ref{chap:#1}
}

\newcommand{\chapterlabel}[1]
{
	\label{chap:#1}
}

\newcommand{\labeledchapter}[2]
{
	\chapter{#1}
	\chapterlabel{#2}
}

\newcommand{\sectionref}[1]
{
	Section \ref{sec:#1}
}

\newcommand{\sectionlabel}[1]
{
	\label{sec:#1}
}

\newcommand{\labeledsection}[2]
{
	\section{#1}
	\sectionlabel{#2}
}

\newcommand{\figureref}[1]
{
	Figure \ref{fig:#1}
}

\newcommand{\figurelabel}[1]
{
	\label{fig:#1}
}

\newcommand{\tableref}[1]
{
	Table \ref{tab:#1}
}

\newcommand{\equationref}[1]
{
	Equation \ref{eq:#1}
}

\newcommand{\equationlabel}[1]
{
	\label{eq:#1}
}

\newcommand{\listingref}[1]
{
	Code Listing \ref{lst:#1}
}

